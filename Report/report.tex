\documentclass[11pt]{article}
\usepackage{a4, fullpage}
\usepackage{bibtopic}
\usepackage{float}
\usepackage{amssymb,amsmath}
\usepackage[T1]{fontenc}
\usepackage{graphicx}
\usepackage{multicol}

\begin{document}

\title{Equivalences Tutor}

\author{Sahil Deepak Jain \\ Supervisor: Fariba Sadri}
\date{\today}

\maketitle
\newpage

% Abstract
\begin{abstract}

\end{abstract}
\newpage

% Table Of Contents
\tableofcontents

\newpage

% Introduction
\section{Introduction}

\emph{Logic} refers to the study of different modes of reasoning conducted or 
assessed according to strict principles of validity. Due to Logic being one of 
the most fundamental aspects of Computer Science, it is taught to every student
pursuing a Computing degree at Imperial College London. During the first term 
in university, every Computing student is taught the Logic course, which aims to
provide the students with knowledge of the syntax and semantics of Propositional 
and Predicate logic. Students can apply this knowledge to complete equivalences
and natural deduction proofs.

A logical system is made up of three things:
\begin{enumerate}
  \item Syntax - this is the formal language specified to express different
         concepts.
  \item Semantics - this is what provides meaning to the language.
  \item Proof theory - this is a way of arguing in the language. This allows us
         to identify valid statements in the language.
\end{enumerate}

In logic, two statements are logically equivalent if they contain the same
logical content. Mendelson stated that "two statements are equivalent if they
have the same truth value in every model." This can be illustrated in the 
following example: \\ \bigskip 

Statement 1: \emph{If Sahil is a final year student, 
then he has to do an individual project} \\ \bigskip 

Statement 2: \emph{If Sahil is not 
doing an individual project, then he is not a final year student} \\ \bigskip 

As we can see, both statements have the same result in same models. When two logic 
statements are equivalent, they can be derived by each other, with the use of 
equivalences which we know to be true.

At the moment, there are a two programs which specifically built to support the
Logic course at Imperial College London. These are:
\begin{enumerate}
  \item Pandora - learning support tool designed to guide the construction of
         natural deduction proofs. 
  \item LOST - application which helps in the learning of logic semantics.
\end{enumerate}

There is no current application which lets the students learn and practice
equivalences. 

\subsection{Report Structure}
\newpage

% Background
\section{Background}

\subsection{Propositional Equivalences}

\subsubsection{And Equivalences}
\begin{enumerate}
  \item A $\land$ B $\equiv$ B $\land$ A \emph{(Commutativity of And)}

\begin{tabular}{| c | c | c |}
  \hline
  A & B & A $\land$ B \\ \hline
  0 & 0 & 0 \\
  0 & 1 & 0 \\
  1 & 0 & 0 \\
  1 & 1 & 1 \\ \hline
\end{tabular}
\quad
\begin{tabular}{| c | c | c |}
  \hline
  A & B & B $\land$ A \\ \hline
  0 & 0 & 0 \\
  0 & 1 & 0 \\
  1 & 0 & 0 \\
  1 & 1 & 1 \\ \hline
\end{tabular}

  \item A $\land$ A $\equiv$ A \emph{(Idempotence of And)}

\begin{tabular}{| c | c |}
  \hline
  A & A $\land$ A \\ \hline
  0 & 0 \\
  0 & 0 \\
  1 & 1 \\
  1 & 1 \\ \hline
\end{tabular}

  \item (A $\land$ B) $\land$ C $\equiv$ A $\land$ (B $\land$ C) \emph{(Associaticity of And)}

\begin{tabular}{| c | c | c | c |}
  \hline
  A & B & C & (A $\land$ B) $\land$ C \\ \hline
  0 & 0 & 0 & 0 \\
  0 & 0 & 1 & 0 \\
  0 & 1 & 0 & 0 \\
  0 & 1 & 1 & 0 \\
  1 & 0 & 0 & 0 \\
  1 & 0 & 1 & 0 \\
  1 & 1 & 0 & 0 \\
  1 & 1 & 1 & 1 \\ \hline
\end{tabular}
\quad
\begin{tabular}{| c | c | c | c |}
  \hline
  A & B & C & A $\land$ (B $\land$ C) \\ \hline
  0 & 0 & 0 & 0 \\
  0 & 0 & 1 & 0 \\
  0 & 1 & 0 & 0 \\
  0 & 1 & 1 & 0 \\
  1 & 0 & 0 & 0 \\
  1 & 0 & 1 & 0 \\
  1 & 1 & 0 & 0 \\
  1 & 1 & 1 & 1 \\ \hline
\end{tabular}
\end{enumerate}

\subsubsection{Or Equivalences}
\begin{enumerate}

  \item A $\lor$ B $\equiv$ B $\lor$ A \emph{(Commutativity of Or)}

\begin{tabular}{| c | c | c |}
  \hline
  A & B & A $\lor$ B \\ \hline
  0 & 0 & 0 \\
  0 & 1 & 1 \\
  1 & 0 & 1 \\
  1 & 1 & 1 \\ \hline
\end{tabular}
\quad
\begin{tabular}{| c | c | c |}
  \hline
  A & B & B $\lor$ A \\ \hline
  0 & 0 & 0 \\
  0 & 1 & 1 \\
  1 & 0 & 1 \\
  1 & 1 & 1 \\ \hline
\end{tabular}

\item A $\lor$ A $\equiv$ A \emph{(Idempotence of Or)}

\begin{tabular}{| c | c |}
  \hline
  A & A $\lor$ A \\ \hline
  0 & 0 \\
  0 & 0 \\
  1 & 1 \\
  1 & 1 \\ \hline
\end{tabular}

  \item (A $\lor$ B) $\lor$ C $\equiv$ A $\lor$ (B $\lor$ C) \emph{(Associaticity of Or)}

\begin{tabular}{| c | c | c | c |}
  \hline
  A & B & C & (A $\lor$ B) $\lor$ C \\ \hline
  0 & 0 & 0 & 0 \\
  0 & 0 & 1 & 1 \\
  0 & 1 & 0 & 1 \\
  0 & 1 & 1 & 1 \\
  1 & 0 & 0 & 1 \\
  1 & 0 & 1 & 1 \\
  1 & 1 & 0 & 1 \\
  1 & 1 & 1 & 1 \\ \hline
\end{tabular}
\quad
\begin{tabular}{| c | c | c | c |}
  \hline
  A & B & C & A $\lor$ (B $\lor$ C) \\ \hline
  0 & 0 & 0 & 0 \\
  0 & 0 & 1 & 1 \\
  0 & 1 & 0 & 1 \\
  0 & 1 & 1 & 1 \\
  1 & 0 & 0 & 1 \\
  1 & 0 & 1 & 1 \\
  1 & 1 & 0 & 1 \\
  1 & 1 & 1 & 1 \\ \hline
\end{tabular}

\end{enumerate}

\subsubsection{Not Equivalences}
\begin{enumerate}

  \item A $\equiv$ $\neg$$\neg$A

\begin{tabular}{| c | c | c |}
  \hline
  A & $\neg$A & $\neg$$\neg$A \\ \hline
  0 & 1 & 0 \\
  1 & 0 & 1 \\ \hline
\end{tabular}

\end{enumerate}

\subsubsection{Implies Equivalences}
\begin{enumerate}

\end{enumerate}

\subsubsection{If and Only If Equivalences}
\begin{enumerate}

\end{enumerate}

\subsubsection{De Morgan Equivalences}
\begin{enumerate}

\end{enumerate}

\subsubsection{Distributibivity of And and Or}
\begin{enumerate}

\end{enumerate}

\newpage

% Project Plan
\section{Project Plan}

\newpage

%Evaluation Plan
\section{Evaluation Plan}

\newpage
\end{document}
