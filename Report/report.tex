\documentclass[12pt]{article}
\usepackage{a4, fullpage}
\usepackage{bibtopic}
\usepackage{float}
\usepackage{amssymb,amsmath}
\usepackage[T1]{fontenc}
\usepackage{graphicx}
\usepackage{multicol}
\usepackage{alltt}

\begin{document}

\title{Equivalences Tutor\\BEng Final Report}

\author{Sahil Deepak Jain\\Department of Computing\\Imperial College London\\Supervisor: Dr. Fariba Sadri}
\date{\today}
\maketitle
\newpage

% Abstract

\begin{abstract}

\end{abstract}
\newpage

% Table Of Contents
\tableofcontents

\newpage
% Introduction
\section{Introduction}

\emph{Logic} refers to the study of different modes of reasoning conducted or 
assessed according to strict principles of validity. Due to Logic being one of 
the most fundamental aspects of Computer Science, it is taught to every student
pursuing a Computing degree at Imperial College London. During the first term 
in university, every Computing student is taught the Logic course, which aims to
provide the students with knowledge of the syntax and semantics of Propositional 
and Predicate logic. Students can apply this knowledge to complete equivalences
and natural deduction proofs.

A logical system is made up of three things:

\begin{enumerate}
  \item Syntax - this is the formal language specified to express different
         concepts.
  \item Semantics - this is what provides meaning to the language.
  \item Proof theory - this is a way of arguing in the language. This allows us
         to identify valid statements in the language.
\end{enumerate}

In logic, two statements are logically equivalent if they contain the same
logical content. Mendelson stated that "two statements are equivalent if they
have the same truth value in every model." This can be illustrated in the 
following example: \\ \bigskip 
Statement 1: \emph{If Sahil is a final year student, 
then he has to do an individual project} \\ \bigskip 
Statement 2: \emph{If Sahil is not 
doing an individual project, then he is not a final year student} \\ \bigskip 
As we can see, both statements have the same result in same models. When two logic 
statements are equivalent, they can be derived by each other, with the use of 
equivalences which we know to be true.

At the moment, there are a two programs which specifically built to support the
Logic course at Imperial College London. These are:

\begin{enumerate}
  \item Pandora - learning support tool designed to guide the construction of
         natural deduction proofs. 
  \item LOST - application which helps in the learning of logic semantics.
\end{enumerate}

There is no current application which lets the students learn and practice
equivalences.

\subsection{Requirements}

The aim of this project is to provide a web based equivalence tutorial tool
with the following features:

\begin{itemize}

	\item Input start and end states of an equivalence, and perform the
				equivalence.
	\item Choose a difficulty of equivalence and practice an auto-generated
				level.
	\item Choose between a logic tree view and a logic expression view while
				completing a level.
	\item Give the student the ability to view previously completed levels.
	\item Give the student the ability to save uncompleted levels, which they
				can complete at a later time.
	\item Automatically suggest a higher level to the student if they are
				progressing well.
\end{itemize}

\subsection{Report Structure}

\newpage


% Background
\section{Background}
\subsection{Propositional Equivalences}

The first thing I had to look at were the logic equivalences which the students
are taught in their respective logic courses. These equivalences are described
in the next few subsections.

\subsubsection{And Equivalences}
\begin{enumerate}

  \item A $\land$ B $\equiv$ B $\land$ A \emph{(Commutativity of And)}

\begin{center}
  \begin{tabular}{| c | c | c | c |}
    \hline
    A & B & A $\land$ B & B $\land$ A \\ \hline
    0 & 0 & 0 & 0 \\
    0 & 1 & 0 & 0 \\
    1 & 0 & 0 & 0 \\
    1 & 1 & 1 & 1 \\ \hline
  \end{tabular}
\end{center}

  \item A $\land$ A $\equiv$ A \emph{(Idempotence of And)}

\begin{center}
  \begin{tabular}{| c | c |}
    \hline
    A & A $\land$ A \\ \hline
    0 & 0 \\
    0 & 0 \\
    1 & 1 \\
    1 & 1 \\ \hline
  \end{tabular}
\end{center}

  \item (A $\land$ B) $\land$ C $\equiv$ A $\land$ (B $\land$ C) \emph{(Associaticity of And)}

\begin{center}
  \begin{tabular}{| c | c | c | c | c |}
    \hline
    A & B & C & (A $\land$ B) $\land$ C & A $\land$ (B $\land$ C) \\ \hline
    0 & 0 & 0 & 0 & 0 \\
    0 & 0 & 1 & 0 & 0 \\
    0 & 1 & 0 & 0 & 0 \\
    0 & 1 & 1 & 0 & 0 \\
    1 & 0 & 0 & 0 & 0 \\
    1 & 0 & 1 & 0 & 0 \\
    1 & 1 & 0 & 0 & 0 \\
    1 & 1 & 1 & 1 & 1 \\ \hline
  \end{tabular}
\end{center}
\end{enumerate}

\subsubsection{Or Equivalences}
\begin{enumerate}

  \item A $\lor$ B $\equiv$ B $\lor$ A \emph{(Commutativity of Or)}

\begin{center}
  \begin{tabular}{| c | c | c | c |}
    \hline
    A & B & A $\lor$ B & B $\lor$ A\\ \hline
    0 & 0 & 0 & 0 \\
    0 & 1 & 1 & 1 \\
    1 & 0 & 1 & 1 \\
    1 & 1 & 1 & 1 \\ \hline
  \end{tabular}
\end{center}

\item A $\lor$ A $\equiv$ A \emph{(Idempotence of Or)}

\begin{center}
  \begin{tabular}{| c | c |}
    \hline
    A & A $\lor$ A \\ \hline
    0 & 0 \\
    0 & 0 \\
    1 & 1 \\
    1 & 1 \\ \hline
  \end{tabular}
\end{center}

  \item (A $\lor$ B) $\lor$ C $\equiv$ A $\lor$ (B $\lor$ C) \emph{(Associaticity of Or)}

\begin{center}
  \begin{tabular}{| c | c | c | c | c |}
    \hline
    A & B & C & (A $\lor$ B) $\lor$ C & A $\lor$ (B $\lor$ C) \\ \hline
    0 & 0 & 0 & 0 & 0 \\
    0 & 0 & 1 & 1 & 1 \\
    0 & 1 & 0 & 1 & 1 \\
    0 & 1 & 1 & 1 & 1 \\
    1 & 0 & 0 & 1 & 1 \\
    1 & 0 & 1 & 1 & 1 \\
    1 & 1 & 0 & 1 & 1 \\
    1 & 1 & 1 & 1 & 1 \\ \hline
  \end{tabular}
\end{center}
\end{enumerate}

\subsubsection{Not Equivalences}
\begin{enumerate}

  \item A $\equiv$ $\neg$$\neg$A

\begin{center}
  \begin{tabular}{| c | c | c |}
    \hline
    A & $\neg$A & $\neg$$\neg$A \\ \hline
    0 & 1 & 0 \\
    1 & 0 & 1 \\ \hline
  \end{tabular}
\end{center}

\end{enumerate}

\subsubsection{Implies Equivalences}
\begin{enumerate}

  \item A $\to$ B $\equiv$ $\neg$A $\lor$ B

\begin{center}
  \begin{tabular}{| c | c | c | c | c |}
    \hline
    A & B & $\neg$A & A $\to$ B & $\neg$A $\lor$ B \\ \hline
    0 & 0 & 1 & 1 & 1 \\
    0 & 1 & 1 & 1 & 1 \\ 
    1 & 0 & 0 & 0 & 0 \\
    1 & 1 & 0 & 1 & 1 \\ \hline
  \end{tabular}
\end{center}

  \item A $\to$ B $\equiv$ $\neg$(A $\land$ $\neg$B)

\begin{center}
  \begin{tabular}{| c | c | c | c | c | c |}
    \hline
    A & B & $\neg$B & A $\land$ $\neg$B & A $\to$ B & $\neg$(A $\land$ $\neg$B) \\ \hline
    0 & 0 & 1 & 0 & 1 & 1 \\
    0 & 1 & 0 & 0 & 1 & 1 \\
    1 & 0 & 1 & 1 & 0 & 0 \\
    1 & 1 & 0 & 0 & 1 & 1 \\ \hline
  \end{tabular}
\end{center}

  \item $\neg$(A $\to$ B) $\equiv$ A $\land$ $\neg$B

\begin{center}
  \begin{tabular}{| c | c | c | c | c | c |}
    \hline
    A & B & $\neg$B & A $\to$ B & $\neg$(A $\to$ B) & A $\land$ $\neg$B \\ \hline
    0 & 0 & 1 & 1 & 0 & 0 \\
    0 & 1 & 0 & 1 & 0 & 0 \\
    1 & 0 & 1 & 0 & 1 & 1 \\
    1 & 1 & 0 & 1 & 0 & 0 \\ \hline
  \end{tabular}
\end{center}
\end{enumerate}

\subsubsection{If and Only If Equivalences}
\begin{enumerate}

  \item A $\leftrightarrow$ B $\equiv$ (A $\to$ B) $\land$ (B $\to$ A)

\begin{center}
  \begin{tabular} {| c | c | c | c | c | c |}
    \hline
    A & B & A $\to$ B & B $\to$ A & A $\leftrightarrow$ B & (A $\to$ B) $\land$ (B $\to$ A) \\ \hline
    0 & 0 & 1 & 1 & 1 & 1 \\
    0 & 1 & 1 & 0 & 0 & 0 \\
    1 & 0 & 0 & 1 & 0 & 0 \\
    1 & 1 & 1 & 1 & 1 & 1 \\ \hline
  \end{tabular} 
\end{center}

  \item A $\leftrightarrow$ B $\equiv$ (A $\land$ B) $\lor$ ($\neg$A $\land$ $\neg$B)

\begin{center}
  \begin{tabular} {| c | c | c | c | c | c | c |}
    \hline
    A & B & A $\land$ B & $\neg$A & $\neg$B & A $\leftrightarrow$ B & (A $\land$ B) $\lor$ ($\neg$A $\land$ $\neg$B) \\ \hline
    0 & 0 & 0 & 1 & 1 & 1 & 1 \\
    0 & 1 & 0 & 1 & 0 & 0 & 0 \\
    1 & 0 & 0 & 0 & 1 & 0 & 0 \\
    1 & 1 & 1 & 0 & 0 & 1 & 1 \\ \hline
  \end{tabular} 
\end{center}
\end{enumerate}

\subsubsection{De Morgan Equivalences}

\begin{enumerate}

  \item $\neg$(A $\land$ B) $\equiv$ $\neg$A $\lor$ $\neg$B

\begin{center}
  \begin{tabular}{| c | c | c | c |}
    \hline
    A & B & $\neg$(A $\land$ B) & $\neg$A $\lor$ $\neg$B \\ \hline
    0 & 0 & 1 & 1 \\
    0 & 1 & 1 & 1 \\ 
    1 & 0 & 1 & 1 \\
    1 & 1 & 0 & 0 \\ \hline
  \end{tabular}
\end{center}

  \item $\neg$(A $\lor$ B) $\equiv$ $\neg$A $\land$ $\neg$B

\begin{center}
  \begin{tabular}{| c | c | c | c |}
    \hline
    A & B & $\neg$(A $\lor$ B) & $\neg$A $\land$ $\neg$B \\ \hline
    0 & 0 & 1 & 1 \\
    0 & 1 & 0 & 0 \\ 
    1 & 0 & 0 & 0 \\
    1 & 1 & 0 & 0 \\ \hline
  \end{tabular}
\end{center}
\end{enumerate}

\subsubsection{Distributibivity of And and Or}
\begin{enumerate}

  \item A $\land$ (B $\lor$ C) $\equiv$ (A $\land$ B) $\lor$ (A $\land$ C)

\begin{center}
  \begin{tabular}{| c | c | c | c | c |}
    \hline
    A & B & C & A $\land$ (B $\lor$ C) & (A $\land$ B) $\lor$ (A $\land$ C) \\ \hline
    0 & 0 & 0 & 0 & 0 \\
    0 & 0 & 1 & 0 & 0 \\
    0 & 1 & 0 & 0 & 0 \\
    0 & 1 & 1 & 0 & 0 \\
    1 & 0 & 0 & 0 & 0 \\
    1 & 0 & 1 & 1 & 1 \\
    1 & 1 & 0 & 1 & 1 \\
    1 & 1 & 1 & 1 & 1 \\ \hline
  \end{tabular}
\end{center}

  \item A $\lor$ (B $\land$ C) $\equiv$ (A $\lor$ B) $\land$ (A $\lor$ C)

\begin{center}
  \begin{tabular}{| c | c | c | c | c |}
    \hline
    A & B & C & A $\lor$ (B $\land$ C) & (A $\lor$ B) $\land$ (A $\lor$ C) \\ \hline
    0 & 0 & 0 & 0 & 0 \\
    0 & 0 & 1 & 0 & 0 \\
    0 & 1 & 0 & 0 & 0 \\
    0 & 1 & 1 & 1 & 1 \\
    1 & 0 & 0 & 1 & 1 \\
    1 & 0 & 1 & 1 & 1 \\
    1 & 1 & 0 & 1 & 1 \\
    1 & 1 & 1 & 1 & 1 \\ \hline
  \end{tabular}
\end{center}
\end{enumerate}
\subsection{Equivalence Examples}
\begin{enumerate}

  \item $\neg$(p $\to$ q) $\equiv$ p $\land$ $\neg$q
  \begin{enumerate}
    \item $\neg$(p $\to$ q)
    \item $\neg$($\neg$p $\lor$ q) \hfill a, Implication rule
    \item $\neg$$\neg$p $\land$ $\neg$q \hfill b, De Morgan's law
    \item p $\land$ $\neg$q \hfill c, Double Negation rule
  \end{enumerate}
  
  \item $\neg$((p $\land$ q) $\to$ r) $\equiv$ (p $\land$ q) $\land$ $\neg$r
  \begin{enumerate}
    \item $\neg$((p $\land$ q) $\to$ r)
    \item $\neg$($\neg$(p $\land$ q) $\lor$ r) \hfill a, Implication rule
    \item $\neg$$\neg$(p $\land$ q) $\land$ $\neg$ r \hfill b, De Morgan's law
    \item (p $\land$ q) $\land$ $\neg$r \hfill c, Double Negation rule
  \end{enumerate}

  \item $\neg$(p $\lor$ $\neg$q) $\lor$ ($\neg$p $\land$ $\neg$q) $\equiv$ $\neg$p
  \begin{enumerate}
    \item $\neg$(p $\lor$ $\neg$q) $\lor$ ($\neg$p $\land$ $\neg$q)
    \item $\neg$(p $\lor$ $\neg$q) $\lor$ $\neg$(p $\land$ q) \hfill a, De Morgan's law
    \item $\neg$[(p $\lor$ $\neg$q) $\land$ (p $\land$ q)] \hfill b, De Morgan's law
    \item $\neg$[p $\lor$ ($\neg$q $\land$ q)] \hfill c, Distributivity
    \item $\neg$(p $\lor$ $\bot$) \hfill d, And rule
    \item $\neg$p \hfill e, Or rule
  \end{enumerate}
\end{enumerate}

\newpage

% Project Plan
\section{Project Plan}
As we are building a completely new application, we have to start from the
very basics of what to do. The inital stages of the project consisted of the
following:
\begin{enumerate}
	\item Meeting up with project supervisor to discuss what she would like as
				 an end result of the project.
	\item Reading through equivalence notes from previous years.
	\item Talk to a few first year students to know where they found problems
				 with equivalences.
	\item Research about the best way to model logic expressions so they could
   			 be easily manipulated in future development.
\end{enumerate}

Once these steps had been carried out, I had a rough idea of where to begin 
the development aspect of the application. After thorough research, I had realised
that the best way to model the logic expressions was to use a tree structure, which
was suggested by Fariba as well. To create the tree structure, I chose to use
ANTLR. This stage consisted of:

\begin{enumerate}
	\item Writing the Lexer rules.
	\item Writing the Parser rules.
	\item Research more about the Abstract Syntax Tree (AST) returned by ANTLR.
\end{enumerate}

After this further research, I realised that using the ANTLR generated ASTs would
not be the most convenient tree structure to use in the long run, so I decided
to create a tree structure of my own.

Future steps include:
\begin{enumerate}
	\item Designing the GUI
	\item Developing the GUI
	\item Writing the equivalence rules which alter the tree structure
	\item Writing the logic expression generator
	\item Writing the progress tracker
  \item Extending the application to support predicate equivalences
\end{enumerate}
\newpage

%Evaluation Plan
\section{Evaluation Plan}

\newpage

\section{Appendix}

\subsection{Lexer Grammar}

\begin{verbatim}
lexer grammar LogicLexer;

options {
  language = Java;
}

@header {
  package eqtutor;
}

WHITESPACE:			( '\t' | ' ' | '\r' | '\n' | '\u000C' )+ { $channel = HIDDEN; };

AND	:						'&';
OR	:						'|';
IFTHEN	:				'->';
IFF 	:					'<>';
NOT 	:					'!';

LPAREN  :				'(';
RPAREN  :				')';

ID	:						('A'..'Z'|'a'..'z') ('A'..'Z'|'a'..'z'|'_')*;
\end{verbatim}

\newpage

\subsection{Parser Grammar}

\begin{verbatim}
parser grammar LogicParser;

options {
  tokenVocab = LogicLexer;
}

@header {
  package eqtutor;

  import java.util.LinkedList;
  import java.util.List;
  import AST.*;
}

@members {
  private boolean hasFoundError = false;

  public void displayRecognitionError(String[] tokenNames, RecognitionException e) {
    hasFoundError = true;
  }

  public boolean hasFoundError() {
    return this.hasFoundError;
  }
}

program returns [AST tree]
  : e = iffexpr {$tree = new AST(new ASTProgramNode($e.node));} EOF
  ;

iffexpr returns [ASTPropositionalNode node]
  : ifthen = ifexpr {$node = $ifthen.node;} 
	  (IFF iff = iffexpr {$node = new ASTIffNode($ifthen.node, $iff.node);})*
  ;
  	
ifexpr returns [ASTPropositionalNode node]
  : or = orexpr {$node = $or.node;} 
		(IFTHEN ifthen = ifexpr {$node = new ASTIfThenNode($or.node, $ifthen.node);})*
  ;

orexpr returns [ASTPropositionalNode node]
  : and = andexpr {$node = $and.node;} 
		(OR or = orexpr {$node = new ASTOrNode($and.node, $or.node);})*
  ;

andexpr returns [ASTPropositionalNode node]
  : not = notexpr {$node = $not.node;} 
		(AND and = andexpr {$node = new ASTAndNode($not.node, $and.node);})*
  ;

notexpr returns [ASTPropositionalNode node]
  : NOT not = notexpr {$node = new ASTNotNode($not.node);}
  | id = identifier {$node = $id.node;}
  ;

identifier returns [ASTPropositionalNode node]
  : ID {$node = new ASTIdentifierNode($ID.text);}
  | LPAREN iffexpr RPAREN {$node = $iffexpr.node;}
  ;
\end{verbatim}

\end{document}
