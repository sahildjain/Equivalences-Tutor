\documentclass[11pt]{article}
\usepackage{a4, fullpage}
\usepackage[inner=4cm,outer=4cm,bottom=2cm]{geometry}
\usepackage{bibtopic}

\begin{document}

\title{Equivalences Tutor}

\author{Sahil Deepak Jain \\ Supervisor: Fariba Sadri}
\date{\today}

\maketitle
\newpage

% Abstract
\begin{abstract}

\end{abstract}
\newpage

% Table Of Contents
\tableofcontents

\newpage

% Introduction
\section{Introduction}
\emph{Logic} refers to the study of different modes of reasoning conducted or
assessed according to strict principles of validity. Due to Logic being one of 
the most fundamental aspects of Computer Science, it is taught to every student
pursuing a Computing degree at Imperial College London. During the first term 
in university, every Computing student is taught the Logic course, which aims to
provide the students with knowledge of the syntax and semantics of Propositional 
and Predicate logic. Students can apply this knowledge to complete equivalences
and natural deduction proofs.

In logic, two statements are logically equivalent if they contain the same
logical content. Mendelson stated that "two statements are equivalent if they
have the same truth value in every model." This can be illustrated in the 
following example:

Statement 1: \emph{If Sahil is a final year student, he has to do an individual
project}

Statement 2: \emph{If Sahil is not a final year student, he does not have to do
an individual project}



\newpage

% Background
\section{Background}
\newpage

% Project Plan
\section{Project Plan}

\newpage

%Evaluation Plan
\section{Evaluation Plan}

\newpage
\end{document}
