\documentclass{report}

\usepackage{a4, fullpage}

\usepackage{graphicx}

\usepackage{alltt}

\usepackage{listings}

\usepackage{color}



\definecolor{dkgreen}{rgb}{0,0.6,0}

\definecolor{gray}{rgb}{0.5,0.5,0.5}

\definecolor{mauve}{rgb}{0.58,0,0.82}

\linespread{1.3}



\lstset{frame=tb,

  language=Java,

  aboveskip=3mm,

  belowskip=3mm,

  showstringspaces=false,

  columns=flexible,

  basicstyle={\small\ttfamily},

  numbers=none,

  numberstyle=\tiny\color{gray},

  keywordstyle=\color{blue},

  commentstyle=\color{dkgreen},

  stringstyle=\color{mauve},

  breaklines=true,

  breakatwhitespace=true

  tabsize=3

}

\begin{document}



%-------------------------------------------------------------------------------

% TITLE PAGE

%-------------------------------------------------------------------------------



\begin{titlepage}

\newcommand{\HRule}{\rule{\linewidth}{0.5mm}}

\center

\textsc{\LARGE Imperial College London} \\[0.5cm]

\textsc{\Large Department of Computing} \\[0.5cm]

\textsc{\large Individual Project} \\[1.5cm]

\HRule \\[0.3cm]

{\huge \bfseries Equivalences Tutor} \\[0.3cm]

\HRule \\[1.5cm]



% author and supervisors

\begin{minipage}{0.4\textwidth}

\begin{flushleft} \large \emph{Author:} \\

Sahil \textsc{Jain}

\end{flushleft}

\end{minipage}~

\begin{minipage}{0.4\textwidth}

\begin{flushright} \large \emph{Supervisor:} \\

Fariba \textsc{Sadri}

\end{flushright}

\begin{flushright} \large \emph{Second marker:} \\

Ian \textsc{Hodkinson}

\end{flushright}

\end{minipage}\\[4cm]

{\large \today}\\[3cm]

\vfill

\end{titlepage}



%-------------------------------------------------------------------------------

% ABSTRACT

%-------------------------------------------------------------------------------



\begin{abstract}

% TODO

\end{abstract}





%-------------------------------------------------------------------------------

% ACKNOWLEDGEMENTS

%-------------------------------------------------------------------------------



\subsection*{\centering Acknowledgements}

I would like to thank Dr. Fariba Sadri for her continuous support and guidance

throughout the course of this project. I would also like to thank Dr. Ian

Hodkinson for his value feedback on my interim report.



Above all, I would like to thank my family for being there for me and giving

me constant support through my education.



%-------------------------------------------------------------------------------

% TABLE OF CONTENTS

%-------------------------------------------------------------------------------



\tableofcontents



%-------------------------------------------------------------------------------

% INTRODUCTION

%-------------------------------------------------------------------------------



\chapter{Introduction}

% what is the project about

% what did I set out to achieve, relevance, importance, difficulty

% key aspects, non-technical



\emph{Logic} refers to the study of different modes of reasoning conducted or 

assessed according to strict principles of validity. Due to Logic being one of 

the most fundamental aspects of Computer Science, it is taught to every student

pursuing a Computing degree at Imperial College London. During the first term 

in university, every Computing student is taught the Logic course, which aims to

provide the students with knowledge of the syntax and semantics of Propositional 

and Predicate logic. Students can apply this knowledge to complete equivalences

and natural deduction proofs.



A logical system is made up of three things:



\begin{enumerate}

  \item Syntax - this is the formal language specified to express different

         concepts.

  \item Semantics - this is what provides meaning to the language.

  \item Proof theory - this is a way of arguing in the language. This allows us

         to identify valid statements in the language.

\end{enumerate}



In logic, two statements are logically equivalent if they contain the same

logical content. Mendelson stated that "two statements are equivalent if they

have the same truth value in every model." This can be illustrated in the 

following example: \\ \bigskip 

Statement 1: \emph{If Sahil is a final year student, 

then he has to do an individual project} \\ \bigskip 

Statement 2: \emph{If Sahil is not 

doing an individual project, then he is not a final year student} \\ \bigskip 

As we can see, both statements have the same result in same models. When two logic 

statements are equivalent, they can be derived by each other, with the use of 

equivalences which we know to be true.





%-------------------------------------------------------------------------------

% BACKGROUND

%-------------------------------------------------------------------------------



\chapter{Background} % 10-20 pages



This section summarises the necessary background knowledge for understanding

this report. It also introduces the concepts of Logic Equivalences.



\section{Propositional Logic}

Before we can define what Propositional Logic is, we have to define what a 

proposition is. Essentially, a proposition is a statement which is either

true or false. This then leads to defining Propositonal Logic as a branch of

symbolic logic dealing with propositions as units and with their combinations

and the connectives that relate them. [1]



\subsection{Syntax}

This is the formal language used in propositional logic. There are three different

parts which make up the syntax of propositional logic; atoms, connectives and

punctuation.



A propositional atom, or propositional variable, is a symbol which has a truth

value, either true or false. Usually, atoms are denoted by letters p, q, r, s, etc.



Connectives are boolean operations which are applied to atoms. There are a total

of seven connectives:



\begin{itemize}

	\item And - written as $\land$. Takes two arguments.

	\item Or - written as $\lor$. Takes two arguments.

	\item Not - written as $\neg$. Takes one argument.

	\item If then - written as $\to$. Takes two arguments.

	\item If and only if - written as $\leftrightarrow$. Takes two arguments.

	\item Truth - written as $\top$. Takes zero arguments.

	\item Falsity - written as $\bot$. Takes zero arguments.

\end{itemize}



\subsection{Semantics}

This is the meaning of a logic formula. As each atom has a truth value, there can

be several combinations of truth values for each atom. To be more specific, if there

are n atoms in a logic formula, then there are a total of $2 ^ n$ combinations of

truth values. Each combination is known as a situation.



Each connective has a different meaning. These meanings can be shown using a truth

table and using the atoms p and q. 



\subsubsection{And}



\begin{center}

  \begin{tabular}{| c | c | c |}

    \hline

    p & q & p $\land$ q \\ \hline

    0 & 0 & 0 \\

    0 & 1 & 0 \\

    1 & 0 & 0 \\

    1 & 1 & 1 \\ \hline

  \end{tabular}

\end{center}



As we can see, p $\land$ q is only true when both p and q are true. Otherwise,

p $\land$ q is always false.



\subsubsection{Or}



\begin{center}

  \begin{tabular}{| c | c | c |}

    \hline

    p & q & p $\lor$ q \\ \hline

    0 & 0 & 0 \\

    0 & 1 & 1 \\

    1 & 0 & 1 \\

    1 & 1 & 1 \\ \hline

  \end{tabular}

\end{center}



As we can see, p $\lor$ q is true when either p or q is true. When both p and q

are false, p $\lor$ q is false as well.



\subsubsection{Not}



\begin{center}

  \begin{tabular}{| c | c |}

    \hline

    p & $\neg$p \\ \hline

    0 & 1 \\

    1 & 0 \\ \hline

  \end{tabular}

\end{center}



The truth value of $\neg$p is always the opposite of p.



\subsubsection{If Then}



\begin{center}

  \begin{tabular}{| c | c | c |}

    \hline

    p & q & p $\to$ q \\ \hline

    0 & 0 & 1 \\

    0 & 1 & 1 \\

    1 & 0 & 0 \\

    1 & 1 & 1 \\ \hline

  \end{tabular}

\end{center}



p $\to$ q is true if p is false or/and if q is true. This means that there is

only one situation where p $\to$ q is false, i.e p is true and q is false. 



\subsubsection{If and Only If}



\begin{center}

  \begin{tabular}{| c | c | c |}

    \hline

    p & q & p $\leftrightarrow$ q \\ \hline

    0 & 0 & 1 \\

    0 & 1 & 0 \\

    1 & 0 & 0 \\

    1 & 1 & 1 \\ \hline

  \end{tabular}

\end{center}



p $\leftrightarrow$ q is true if p and q both have the same truth value.



\subsubsection{Truth and Falsity}

Truth is an atom which is always true, and Falsity is an atom which is always false.



\section{Equivalences}

As we introduced the example of "\emph{If Sahil is a final year student, 

then he has to do an individual project}" being the same as "\emph{If Sahil is not 

doing an individual project, then he is not a final year student}", all logic

formulae are equivalent to other logic formulae.



We can show this by converting the above example into propositional logic. If we let

the atom \emph{p} denote "\emph{Sahil is a final year student}" and let the atom \emph{q}

denote "\emph{Sahil is doing an individual project}", then the statement "\emph{If Sahil 

is a final year student, then he has to do an individual project}" becomes p $\to$ q

and the statement "\emph{If Sahil is not doing an individual project, then he is 

not a final year student}" becomes $\neg$q $\to$ $\neg$p.



If we construct the truth tables of both of these formulae, we get:



\begin{center}

  \begin{tabular}{| c | c | c |}

    \hline

    p & q & p $\to$ q \\ \hline

    0 & 0 & 1 \\

    0 & 1 & 1 \\

    1 & 0 & 0 \\

    1 & 1 & 1 \\ \hline

  \end{tabular}

\end{center}



\begin{center}

  \begin{tabular}{| c | c | c |}

    \hline

    p & q & $\neg$q $\to$ $\neg$p \\ \hline

    0 & 0 & 1 \\

    0 & 1 & 1 \\

    1 & 0 & 0 \\

    1 & 1 & 1 \\ \hline

  \end{tabular}

\end{center}



As we can see, both of the truth tables are the same. This means that in any given

situation, both of the logic formulae have the same truth value, meaning that they

are equivalent.



There are several defined equivalences for each operator.



\subsection{And Equivalences}

\begin{enumerate}



  \item A $\land$ B $\equiv$ B $\land$ A \emph{(Commutativity of And)}

  \item A $\land$ A $\equiv$ A \emph{(Idempotence of And)}

  \item (A $\land$ B) $\land$ C $\equiv$ A $\land$ (B $\land$ C) \emph{(Associaticity of And)}



\end{enumerate}



\subsection{Or Equivalences}

\begin{enumerate}



  \item A $\lor$ B $\equiv$ B $\lor$ A \emph{(Commutativity of Or)}

	\item A $\lor$ A $\equiv$ A \emph{(Idempotence of Or)}

  \item (A $\lor$ B) $\lor$ C $\equiv$ A $\lor$ (B $\lor$ C) \emph{(Associaticity of Or)}



\end{enumerate}



\subsection{Not Equivalences}

\begin{enumerate}



  \item A $\equiv$ $\neg$$\neg$A



\end{enumerate}



\subsection{Implies Equivalences}

\begin{enumerate}



  \item A $\to$ B $\equiv$ $\neg$A $\lor$ B

  \item A $\to$ B $\equiv$ $\neg$(A $\land$ $\neg$B)

  \item $\neg$(A $\to$ B) $\equiv$ A $\land$ $\neg$B



\end{enumerate}



\subsection{If and Only If Equivalences}

\begin{enumerate}



  \item A $\leftrightarrow$ B $\equiv$ (A $\to$ B) $\land$ (B $\to$ A)

  \item A $\leftrightarrow$ B $\equiv$ (A $\land$ B) $\lor$ ($\neg$A $\land$ $\neg$B)



\end{enumerate}



\subsection{De Morgan Equivalences}



\begin{enumerate}



  \item $\neg$(A $\land$ B) $\equiv$ $\neg$A $\lor$ $\neg$B

  \item $\neg$(A $\lor$ B) $\equiv$ $\neg$A $\land$ $\neg$B



\end{enumerate}



\subsection{Distributibivity of And and Or}

\begin{enumerate}



  \item A $\land$ (B $\lor$ C) $\equiv$ (A $\land$ B) $\lor$ (A $\land$ C)

  \item A $\lor$ (B $\land$ C) $\equiv$ (A $\lor$ B) $\land$ (A $\lor$ C)



\end{enumerate}



\section{Examples of Equivalence Problems}



\subsection{Equivalence Examples}

\begin{enumerate}



  \item $\neg$(p $\to$ q) $\equiv$ p $\land$ $\neg$q

  \begin{enumerate}

    \item $\neg$(p $\to$ q)

    \item $\neg$($\neg$p $\lor$ q) \hfill a, Implication rule

    \item $\neg$$\neg$p $\land$ $\neg$q \hfill b, De Morgan's law

    \item p $\land$ $\neg$q \hfill c, Double Negation rule

  \end{enumerate}

  

  \item $\neg$((p $\land$ q) $\to$ r) $\equiv$ (p $\land$ q) $\land$ $\neg$r

  \begin{enumerate}

    \item $\neg$((p $\land$ q) $\to$ r)

    \item $\neg$($\neg$(p $\land$ q) $\lor$ r) \hfill a, Implication rule

    \item $\neg$$\neg$(p $\land$ q) $\land$ $\neg$ r \hfill b, De Morgan's law

    \item (p $\land$ q) $\land$ $\neg$r \hfill c, Double Negation rule

  \end{enumerate}



  \item $\neg$(p $\lor$ $\neg$q) $\lor$ ($\neg$p $\land$ $\neg$q) $\equiv$ $\neg$p

  \begin{enumerate}

    \item $\neg$(p $\lor$ $\neg$q) $\lor$ ($\neg$p $\land$ $\neg$q)

    \item $\neg$(p $\lor$ $\neg$q) $\lor$ $\neg$(p $\land$ q) \hfill a, De Morgan's law

    \item $\neg$[(p $\lor$ $\neg$q) $\land$ (p $\land$ q)] \hfill b, De Morgan's law

    \item $\neg$[p $\lor$ ($\neg$q $\land$ q)] \hfill c, Distributivity

    \item $\neg$(p $\lor$ $\bot$) \hfill d, And rule

    \item $\neg$p \hfill e, Or rule

  \end{enumerate}

\end{enumerate}



\section{Current Logic Tutorial Applications}



In the first year Logic course, several different concepts are taught to the students.

Some of the concepts have applications installed on the lab machines which are produced

to help students understand these concepts better. Firstly, there is an application

called \emph{Pandora}, which is a learning support tool designed to guide the

construction of natural deduction proofs. The second application the department

provides is \emph{LOST}, which helps students understand logic semantics better. 



Equivalences is a large part of the Logic course, but currently, there is no

tutorial application which the students can use to understand how to apply

equivalences better, or which provides a user friendly interface for the students 

to use. 



%-------------------------------------------------------------------------------

% MAIN BODY

%-------------------------------------------------------------------------------



\chapter{Implementation and Features}



\section{Modelling Logic Formulae}



The first part that had to be implemented was modelling logic formulae so that

they could be manipulated when applying equivalences. In the first year Logic course

taught by Ian Hodkinson, the students are taught the Logic formation tree.



The example given in the first year slides is:

% TODO insert the tree from first year slides



Using this formation tree, we can model logical formulae in a tree structure in

Java by creating our own data structure.



\subsection{Lexer}



The first step in modelling the logical formulae into a tree structure is lexical 

analysis. This is the process of converting a sequence of characters into a sequence

of tokens which can be parsed in the future.



As writing a lexer from scratch would have been very time consuming and tedious,

I researched about the tools which had been developed for generating lexers after

inputting the lexical grammar. The best tool which I came across was ANTLR, as it

provides a very easy to use debugger where you can load the generated code and step

through it. 



\subsubsection{Lexical Grammar}

\begin{verbatim}

lexer grammar LogicLexer;



options {

  language = Java;

}



@header {

  package eqtutor;

}



WHITESPACE:			( '\t' | ' ' | '\r' | '\n' | '\u000C' )+ { $channel = HIDDEN; };



AND	:						'&';

OR	:						'|';

IFTHEN	:				'->';

IFF 	:					'<>';

NOT 	:					'!';



LPAREN  :				'(';

RPAREN  :				')';



ID	:						('A'..'Z'|'a'..'z') ('A'..'Z'|'a'..'z'|'_')*;

\end{verbatim}



\subsection{Parser}



Once the formulae has been tokenised by the lexer, it then has to be parsed. Parsing

is the process where the tokens are used to build a data structure, which is usually

a hierarchical structure. This data structure gives a structural representation

of the input.



ANTLR provides both lexer and parser generators. Due to having the ability of doing both

of these steps using one tool which I had become familiar with, I decided to use the

ANTLR parser generator.





\subsubsection{Parser Grammar}



\begin{verbatim}

parser grammar LogicParser;



options {

  tokenVocab = LogicLexer;

}



@header {

  package eqtutor;



  import java.util.LinkedList;

  import java.util.List;

  import AST.*;

}



@members {

  private boolean hasFoundError = false;



  public void displayRecognitionError(String[] tokenNames, RecognitionException e) {

    hasFoundError = true;

  }



  public boolean hasFoundError() {

    return this.hasFoundError;

  }

}



program returns [AST tree]

  : e = iffexpr {$tree = new AST(new ASTProgramNode($e.node));} EOF

  ;



iffexpr returns [ASTPropositionalNode node]

  : ifthen = ifexpr {$node = $ifthen.node;} 

	  (IFF iff = iffexpr {$node = new ASTIffNode($ifthen.node, $iff.node);})*

  ;

  	

ifexpr returns [ASTPropositionalNode node]

  : or = orexpr {$node = $or.node;} 

		(IFTHEN ifthen = ifexpr {$node = new ASTIfThenNode($or.node, $ifthen.node);})*

  ;



orexpr returns [ASTPropositionalNode node]

  : and = andexpr {$node = $and.node;} 

		(OR or = orexpr {$node = new ASTOrNode($and.node, $or.node);})*

  ;



andexpr returns [ASTPropositionalNode node]

  : not = notexpr {$node = $not.node;} 

		(AND and = andexpr {$node = new ASTAndNode($not.node, $and.node);})*

  ;



notexpr returns [ASTPropositionalNode node]

  : NOT not = notexpr {$node = new ASTNotNode($not.node);}

  | id = identifier {$node = $id.node;}

  ;



identifier returns [ASTPropositionalNode node]

  : ID {$node = new ASTIdentifierNode($ID.text);}

  | LPAREN iffexpr RPAREN {$node = $iffexpr.node;}

  ;

\end{verbatim}



\section{Abstract Syntax Tree}



The Abstract Syntax Tree (AST) is the tree representation of the structure

of the logic expression. Each node of the tree represents either a connective

or an identifier of the logic expression.



ANTLR provided an implementation of the AST as well, but I decided not to use it

since manipulation of the trees provided was tough and confusing. Due to this,

I created my own data structure to store the logic expression.



Once the tree is parsed, we obtain an object of type AST. The tree contains

a node of type ASTProgramNode, which is the highest level of the tree. The ASTProgramNode

node contains an ASTPropositionalNode. ASTPropositionalNode is an abstract class

which contains the abstract methods for all of the propositional operators. The

ASTPropositionalNode class is shown below:



\begin{lstlisting}[caption=Methods held in a Propositional Node, label=find]

public abstract class ASTPropositionalNode extends ASTNode {



	/*--Checks if the current node is equal to the input node--*/

	public abstract boolean equals(ASTPropositionalNode node);



	/*--Returns the string representation of the node--*/

	public abstract String toString();



	/*--Returns the string representation of the node which can be parsed--*/

	public abstract String toParserString();



	/*--Returns a tree map of the identifiers in the node--*/

	public abstract TreeMap<String, Integer> numIdentifiers(TreeMap<String, Integer> identifiers);

	

	/*--Returns the value of the node, e.g TRUTH and TRUTH = TRUTH--*/

	public abstract int value(TreeMap<String, Integer> id);



	/*--Creates the JPanel for the node which is used to solve equivalences--*/

	public abstract JPanel createJPanel(NewPersonalEquivalenceListener l, boolean side);

	

	/*--Copy constructor for the node--*/

	public abstract ASTPropositionalNode copy();



}

\end{lstlisting}



\section{Equivalences Data Structure}



The equivalences the user perform have to be used in several different features.

To make it easy to use and manipulate the equivalences, a data structure in which

the equivalences could be stored had to be decided upon. 



Before I could decide on which data structure to use, I had to delve into the details

of what I had to store in order to complete equivalences, and how exactly an

equivalence was done. In an equivalence, you only ever have to use the previous

formula to obtain the next equivalent formula. This is illustrated by the 

example below:



\begin{enumerate}

    \item $\neg$(p $\to$ q)

		\item $\neg$($\neg$(p $\land$ $\neg$q)) \hfill 1, Implication rule

    \item $\neg$($\neg$p $\lor$ q) \hfill 1, Implication rule

    \item $\neg$$\neg$p $\land$ $\neg$q \hfill 3, De Morgan's law

    \item p $\land$ $\neg$q \hfill 4, Double Negation rule

\end{enumerate}



As we can see, line b is unneccesary in this equivalence, although true.

We also notice that line d is obtained using line c, and line e from line d.



Due to the presence of this feature, I considered the use of linked lists

to store equivalences. A linked list is essentially a list implemented by

each item having a link to the next item [3]. In the equivalences case, this

means that every formula would have a link to a logically equivalent formula

which has been derived by applying a single equivalence.



Before I could create the Linked List class, I had to create the structure of

each node. Below are the fields stored in a node:



\begin{lstlisting}[caption=Fields held in a list node, label=find]

/*--line number of the formula--*/

private int lineNumber;



/*--the Abstract Syntax Tree of the formula--*/

private AST tree;



/*--the next node in the equivalence--*/

private EquivalenceLinkNode next;

\end{lstlisting}



Once the node class was written, the list class could be implemented. Below

are the fields and methods in the list class:



\begin{lstlisting}[caption=Methods and fields in the Linked List class, label=find]

public class EquivalenceLinkedList {

	

	/*--the first node of the list--*/

	private EquivalenceLinkNode head;



	/*--the number of nodes in the list--*/

	private int size;



	/*--constructor--*/

	public EquivalenceLinkedList();



	/*--returns true if the list is empty, false otherwise--*/	

	public boolean isEmpty();



	/*--adds a node to the end of the list--*/

	public void add(EquivalenceLinkNode node);

	

	/*--removes the last element of the list--*/

	public void removeLast();



	/*--returns the first element of the list--*/

	public EquivalenceLinkNode getHead();



	/*--sets the parameter node as the head of the list--*/

	public void setHead(EquivalenceLinkNode head);



	/*--returns the size of the list--*/	

	public int getSize();



	/*--sets the parameter size as the size of the list--*/	

	public void setSize(int size);



	/*--returns the last node of the list--*/

	public EquivalenceLinkNode getLast()



}

\end{lstlisting}



\section{Implementation of Equivalences}



As formulae are stored as a tree, to apply an equivalence to a formula, the tree

structure has to be altered. As nodes had to be altered, the tree had to be traversed,

which lead to a few problems. In the method which applied equivalences, which is

discussed later, the node to be changed and the new node were passed in as parameters.

The node was then altered. A problem arose due to the fact that there could be several

instances of the node in the tree. For example, consider the formula (p $\land$ p) $\to$ 

(q $\to$ (p $\land$ p)). Suppose we wanted to apply to equivalence (p $\land$ p) $\equiv$ p to

the first instance of p $\land$ p. The method would initially return the tree with formula 

p $\to$ (q $\to$ p). To overcome this problem, a key had to be designated to each node, so

when an equivalence was applied, the key of the node would be a parameter in the method.

This allows for the equivalence to be correctly applied to the formula.



For each equivalence, there is a different method. An example method is shown below. The

example is for the equivalence  p $\land$ q = $\neg$($\neg$p $\lor$ $\neg$q).



\begin{lstlisting}[caption=Example of an equivalence method, label=equiveg]

public AST deMorgan() {

  try {

    AST tree = getTree();

    int key = getKey();

		

    /*--finds the node with the given key in the tree--*/

    ASTNode node  = find(tree.getRoot(), key);



    /*--only applies the equivalence if the found node is an instance

        of an And node--*/

    if(node instanceof ASTAndNode) {

      ASTAndNode andNode = (ASTAndNode) node;

      ASTPropositionalNode left = andNode.getLeft();

      ASTPropositionalNode right = andNode.getRight();



      /*--creates a not node of the left child of the and node--*/

      ASTNotNode notLeft = new ASTNotNode(tree.getKey(), left);

      tree.setKey(tree.getKey() + 1);



      /*--creates a not node of the right child of the and node--*/

      ASTNotNode notRight = new ASTNotNode(tree.getKey(), right);

      tree.setKey(tree.getKey() + 1);



      /*--creates the or node needed in the equivalence--*/

      ASTOrNode orNode = new ASTOrNode(tree.getKey(), notLeft, notRight);

      tree.setKey(tree.getKey() + 1);



      /*--creates the not node which is equivalence to the initial and node--*/

      ASTNotNode notNode = new ASTNotNode(tree.getKey(), orNode);

      tree.setKey(tree.getKey() + 1);



      /*--replaces the and node with the not node--*/

      ASTPropositionalNode p = replace(tree.getRoot().getLeaf(), notNode, key);

      ASTProgramNode program = tree.getRoot();

      program.setLeaf(p);



      /*--creates and returns the new logically equivalent tree--*/

      AST t = new AST(tree.getKey(), program);

      return t;

    }

  }

  catch(Exception e) {

    return null;

  }

  return null;

}

\end{lstlisting}



The same structure is used for other equivalence methods.



Clearly, the find and replace methods are essential to this method. The find method

is shown in \ref{find}, and the replace method is shown in \ref{replace}



\begin{lstlisting}[caption=Find method for equivalences, label=find]

public static ASTNode find(ASTNode node, int key) {

  if(node.getKey() == key) {

    return node;

  }

  if(node instanceof ASTPropositionalBinaryNode) {

    ASTPropositionalBinaryNode binary = (ASTPropositionalBinaryNode) node;

    ASTNode left = find(binary.getLeft(), key);

    ASTNode right = find(binary.getRight(), key);

    if(left != null) {

      return left;

    }

    if(right != null) {

      return right;

    }

  }

  if(node instanceof ASTPropositionalUnaryNode) {

    ASTPropositionalUnaryNode unary = (ASTPropositionalUnaryNode) node;

    ASTNode ret = find(unary.getLeaf(), key);

    if(ret != null) {

      return ret;

    }

  }

  return null;

}

\end{lstlisting}



\begin{lstlisting}[caption=Replace method for equivalences, label=replace]

public static ASTPropositionalNode replace(ASTPropositionalNode prop, ASTNode node, int key) {

  if(prop.getKey() == key) {

    return (ASTPropositionalNode) node;

  }

  if(prop instanceof ASTPropositionalBinaryNode) {

    ASTPropositionalBinaryNode binary = (ASTPropositionalBinaryNode) prop;

    if(find(binary.getLeft(), key) != null) {

      binary.setLeft(replace(((ASTPropositionalBinaryNode) prop).getLeft(), node, key));

    }

    if(find(((ASTPropositionalBinaryNode) prop).getRight(), key) != null) {

      binary.setRight(replace(((ASTPropositionalBinaryNode) prop).getRight(), node, key));

    }

    return binary;

  }

  if(prop instanceof ASTPropositionalUnaryNode) {

    ASTPropositionalUnaryNode unary = (ASTPropositionalUnaryNode) prop;

    unary.setLeaf(replace(unary.getLeaf(), node, key));

    return unary;

  }

  return null;

}

\end{lstlisting}



\section{Description of Packages}

In this section, a description of each package written for the

implementation of the application is given.



\subsection{AST}

This package contains all of the classes needed for the Logic tree structure.

The top level class is the AST class, which contains all of the data about

the tree, such as the top level node.



For nodes, there is an abstract class called ASTNode, which each node extends.

Every node has a key, so the getter and setter for the keys are abstract methods

in the ASTNode class. Then, there is a ASTPropositionalNode abstract class, which

every propositional node extends. As discussed in the Background section, there

are two different types of propositional nodes; unary and binary. Unary nodes

include Truth, Falsity and Not, and the binary nodes are the Or, And, If then,

If and only if nodes. To accomodate for these two types of propositional nodes,

two abstract classes were creates; ASTPropositionalBinaryNode and

ASTPropositionalUnaryNode. Both of these classes contain respective methods

needed for the manipulation of the trees.



For testing purposes, the Visitor design pattern is used to print the tree

to console. The Visitor design pattern allows the separation of an algorithm

from an object structure on which it operates.



\subsection{gui}

This package contains all of the classes which are needed for the Graphical

User Interface (GUI).



\subsection{database}

This package handles all of the database interaction that occurs in the program.

As we are provided with PostgreSQL databases, I decided on using JDBC to interact

with the database.

Whenever the application makes a query to the database, it has to connect
to it. To make this easy, a class DatabaseAdaptor contains a static method
which connects to the database. Every query and update method calls this
static method.

During the implementation of the application, the database had to be reset
several times. It became tedious to open terminal and clear and then
initialise the database several times. To overcome this problem, a class
DatabaseInitialisor was created, which resets and initialises the database
with one method call.

The other classes in the package contain all necessary methods for queries and
updates to the database. Whenever a query is made, a ResultSet object is returned
to the method. To keep all database interaction away from the main implementation
of the code, the ResultSet is converted to another object which is suitable
for the query.



\subsection{buttonlisteners}



\subsection{dialogs}



\subsection{eqtutor}

This is the package which contains the lexer, parser and logic expression

generator.



\subsection{equivalence}

This package handles all of the equivalences and manipulation of the logic trees.

There is one major abstract class, which all of the Equivalence classes

extend. This class contains the find and replace methods, which have been explained
earlier in this chapter.

For every different type of node, there is a different class which contains
the methods for all of its related equivalences. All of these methods return
the new logically equivalent AST.

This package also contains the data structure of the linked list and link node
which are needed to store an equivalence which the user performs.



\section{Logic Expression Generator}



\section{Equivalence Solvers}



Two different interfaces are implemented which the user can use to solve equivalences.

Both of the modes have several features in common. Firstly, regardless of what mode

the user selects to perform an equivalence, the application is divided into two

different distinct panels. Once the start and end state have been entered, the

start state will be visible on the left panel, and the end state will be available on

the right panel. The advantage of having the interface laid out like this is that

it allows the user to work from the start state and from the end state, apply

equivalences, and eventually meet in the middle. Once the final equivalences on each

side match, the equivalence is complete. When performing these equivalences, the

user can switch modes as they please.



The user also has the ability to save an equivalence which they are doing, either

to file or on a database. The user can undo their last equivalences from both panels

as well.





\subsection{Checking if two formulae are logically equivalent}



When a user starts a new equivalence, he has to first put input the start

state and the end state. These two formulae are then checked to be equivalent

or not, as the user should not be allowed to perform an equivalence problem

if the start and end state are not equivalent.



If we were doing this on paper, we would just construct the truth table for

the two formulae, and then compare them to be equal. The same idea is programmed

in the AST class. As the AST only contains the structure, we have to assign each

identifier with a truth value, so that every situation is accounted for while

creating a truth table.



To achieve this, firstly a Tree Map of the identifiers is obtained. A Tree Map is

used as it sorts the map according to the natural ordering of its keys [2], which are

strings in this case, which means that the keys are stored alphabetically. This means

the order of identifiers in the map of both formulae will be the same, allowing for

easy comparison of truth tables. Once the tree map of the identifiers is obtained,

the size of the map can be used to determine every situation. This is illustrated

by the example below:



Consider the formula p $\land$ q $\land$ r, the situations are:



\begin{center}

  \begin{tabular}{| c | c | c |}

    \hline

    p & q & r\\ \hline

    0 & 0 & 0 \\

    0 & 0 & 1 \\

		0 & 1 & 0 \\

		0 & 1 & 1 \\

		1 & 0 & 0 \\

		1 & 0 & 1 \\

    1 & 1 & 0 \\

    1 & 1 & 1 \\ \hline

  \end{tabular}

\end{center}



If we look at each situation in order, we can see that they are binary numbers

starting from 0, going to $2^n$ where n is the number of identifiers. This

connection can be used to assign values to identifiers. The pseudocode is shown

below:



%TODO write pseudocode

\begin{lstlisting}



public int[][] createTruthTable() {

  TreeMap<String, Integer> identifiers = this.identifiers();

  int numIdentifiers = identifiers.size();

  int temp = (int) Math.pow(2, numIdentifiers);

  int[][] truthTable = new int[temp][numIdentifiers + 1];

  for(int i = 0; i < truthTable.length; i++) {

    String binary = Integer.toBinaryString(i);

    if(binary.length() < numIdentifiers) {

      int numZeroes = numIdentifiers - binary.length();

      String zeroes = "";

      while(zeroes.length() < numZeroes) {

	      zeroes += "0";

      }

      binary = zeroes + binary;

    }

    for(int j = 0; j < truthTable[i].length - 1; j++) {

      char c = binary.charAt(j);

      int n = Integer.parseInt(Character.toString(c));

      truthTable[i][j] = n;

    }

  }

  for(int i = 0; i < truthTable.length; i++) {

    Set<String> keys = identifiers.keySet();

    TreeMap<String, Integer> id = new TreeMap<String, Integer>();

    Iterator<String> it = keys.iterator();

    int j = 0;

    while(it.hasNext()) {

      String key = (String) it.next();

      id.put(key, truthTable[i][j]);

      j++;

    }

    truthTable[i][j] = value(id);

  }

  return truthTable;

}



\end{lstlisting}



\subsection{Easy Equivalence Solver}



The first mode for solving equivalences is the easy mode. In this interface,

the user can click on either an identifier or a connective. Once the user

clicks on an identifier, a dialog opens up with all possible equivalences

that can be applied to the identifier, for example, p $\equiv$ $\neg$$\neg$p.

If the user clicks on an connective instead of an identifier, the dialog which

opens up shows the equivalences that can be applied to the clicked connective,

for example, if $\land$ was clicked from the formula a $\land$ b, then

one of the equivalences that would be seen would be the commutativity

equivalence, i.e a $\land$ b $\equiv$ b $\land$ a.



%TODO add an image of the interface



\subsection{Hard Equivalence Solver}





The hard equivalence solver is an interface for solving equivalences where

the student enters logic expressions into a text field. This expression is

then compared to the previous expression to see if it is logically equivalent,

and to see if only one of the given equivalences has been applied. 



%TODO add an image of the interface







%-------------------------------------------------------------------------------

% EVALUATION

%-------------------------------------------------------------------------------



\chapter{Evaluation}



%-------------------------------------------------------------------------------

% CONCLUSIONS AND FUTURE WORK

%-------------------------------------------------------------------------------



\chapter{Conclusions and Future Work}

\section{Learning outcomes}

\section{Potential improvements}

\section{Potential extensions}



%-------------------------------------------------------------------------------

% BIBLIOGRAPHY

%-------------------------------------------------------------------------------



\begin{thebibliography}{9}

%\bibitem{lamport94}

% Leslie Lamport,

% \emph{\LaTeX: A Document Preparation System}.

% Addison Wesley, Massachusetts,

% 2nd Edition,

% 1994.

\end{thebibliography}



%-------------------------------------------------------------------------------

% APPENDIX

%-------------------------------------------------------------------------------



\appendix

\chapter{Equivalences Truth Tables}

\section{And Equivalences}

\begin{enumerate}



  \item A $\land$ B $\equiv$ B $\land$ A \emph{(Commutativity of And)}



\begin{center}

  \begin{tabular}{| c | c | c | c |}

    \hline

    A & B & A $\land$ B & B $\land$ A \\ \hline

    0 & 0 & 0 & 0 \\

    0 & 1 & 0 & 0 \\

    1 & 0 & 0 & 0 \\

    1 & 1 & 1 & 1 \\ \hline

  \end{tabular}

\end{center}



  \item A $\land$ A $\equiv$ A \emph{(Idempotence of And)}



\begin{center}

  \begin{tabular}{| c | c |}

    \hline

    A & A $\land$ A \\ \hline

    0 & 0 \\

    0 & 0 \\

    1 & 1 \\

    1 & 1 \\ \hline

  \end{tabular}

\end{center}



  \item (A $\land$ B) $\land$ C $\equiv$ A $\land$ (B $\land$ C) \emph{(Associaticity of And)}



\begin{center}

  \begin{tabular}{| c | c | c | c | c |}

    \hline

    A & B & C & (A $\land$ B) $\land$ C & A $\land$ (B $\land$ C) \\ \hline

    0 & 0 & 0 & 0 & 0 \\

    0 & 0 & 1 & 0 & 0 \\

    0 & 1 & 0 & 0 & 0 \\

    0 & 1 & 1 & 0 & 0 \\

    1 & 0 & 0 & 0 & 0 \\

    1 & 0 & 1 & 0 & 0 \\

    1 & 1 & 0 & 0 & 0 \\

    1 & 1 & 1 & 1 & 1 \\ \hline

  \end{tabular}

\end{center}

\end{enumerate}



\section{Or Equivalences}

\begin{enumerate}



  \item A $\lor$ B $\equiv$ B $\lor$ A \emph{(Commutativity of Or)}



\begin{center}

  \begin{tabular}{| c | c | c | c |}

    \hline

    A & B & A $\lor$ B & B $\lor$ A\\ \hline

    0 & 0 & 0 & 0 \\

    0 & 1 & 1 & 1 \\

    1 & 0 & 1 & 1 \\

    1 & 1 & 1 & 1 \\ \hline

  \end{tabular}

\end{center}



\item A $\lor$ A $\equiv$ A \emph{(Idempotence of Or)}



\begin{center}

  \begin{tabular}{| c | c |}

    \hline

    A & A $\lor$ A \\ \hline

    0 & 0 \\

    0 & 0 \\

    1 & 1 \\

    1 & 1 \\ \hline

  \end{tabular}

\end{center}



  \item (A $\lor$ B) $\lor$ C $\equiv$ A $\lor$ (B $\lor$ C) \emph{(Associaticity of Or)}



\begin{center}

  \begin{tabular}{| c | c | c | c | c |}

    \hline

    A & B & C & (A $\lor$ B) $\lor$ C & A $\lor$ (B $\lor$ C) \\ \hline

    0 & 0 & 0 & 0 & 0 \\

    0 & 0 & 1 & 1 & 1 \\

    0 & 1 & 0 & 1 & 1 \\

    0 & 1 & 1 & 1 & 1 \\

    1 & 0 & 0 & 1 & 1 \\

    1 & 0 & 1 & 1 & 1 \\

    1 & 1 & 0 & 1 & 1 \\

    1 & 1 & 1 & 1 & 1 \\ \hline

  \end{tabular}

\end{center}

\end{enumerate}



\section{Not Equivalences}

\begin{enumerate}



  \item A $\equiv$ $\neg$$\neg$A



\begin{center}

  \begin{tabular}{| c | c | c |}

    \hline

    A & $\neg$A & $\neg$$\neg$A \\ \hline

    0 & 1 & 0 \\

    1 & 0 & 1 \\ \hline

  \end{tabular}

\end{center}



\end{enumerate}



\section{Implies Equivalences}

\begin{enumerate}



  \item A $\to$ B $\equiv$ $\neg$A $\lor$ B



\begin{center}

  \begin{tabular}{| c | c | c | c | c |}

    \hline

    A & B & $\neg$A & A $\to$ B & $\neg$A $\lor$ B \\ \hline

    0 & 0 & 1 & 1 & 1 \\

    0 & 1 & 1 & 1 & 1 \\ 

    1 & 0 & 0 & 0 & 0 \\

    1 & 1 & 0 & 1 & 1 \\ \hline

  \end{tabular}

\end{center}



  \item A $\to$ B $\equiv$ $\neg$(A $\land$ $\neg$B)



\begin{center}

  \begin{tabular}{| c | c | c | c | c | c |}

    \hline

    A & B & $\neg$B & A $\land$ $\neg$B & A $\to$ B & $\neg$(A $\land$ $\neg$B) \\ \hline

    0 & 0 & 1 & 0 & 1 & 1 \\

    0 & 1 & 0 & 0 & 1 & 1 \\

    1 & 0 & 1 & 1 & 0 & 0 \\

    1 & 1 & 0 & 0 & 1 & 1 \\ \hline

  \end{tabular}

\end{center}



  \item $\neg$(A $\to$ B) $\equiv$ A $\land$ $\neg$B



\begin{center}

  \begin{tabular}{| c | c | c | c | c | c |}

    \hline

    A & B & $\neg$B & A $\to$ B & $\neg$(A $\to$ B) & A $\land$ $\neg$B \\ \hline

    0 & 0 & 1 & 1 & 0 & 0 \\

    0 & 1 & 0 & 1 & 0 & 0 \\

    1 & 0 & 1 & 0 & 1 & 1 \\

    1 & 1 & 0 & 1 & 0 & 0 \\ \hline

  \end{tabular}

\end{center}

\end{enumerate}



\section{If and Only If Equivalences}

\begin{enumerate}



  \item A $\leftrightarrow$ B $\equiv$ (A $\to$ B) $\land$ (B $\to$ A)



\begin{center}

  \begin{tabular} {| c | c | c | c | c | c |}

    \hline

    A & B & A $\to$ B & B $\to$ A & A $\leftrightarrow$ B & (A $\to$ B) $\land$ (B $\to$ A) \\ \hline

    0 & 0 & 1 & 1 & 1 & 1 \\

    0 & 1 & 1 & 0 & 0 & 0 \\

    1 & 0 & 0 & 1 & 0 & 0 \\

    1 & 1 & 1 & 1 & 1 & 1 \\ \hline

  \end{tabular} 

\end{center}



  \item A $\leftrightarrow$ B $\equiv$ (A $\land$ B) $\lor$ ($\neg$A $\land$ $\neg$B)



\begin{center}

  \begin{tabular} {| c | c | c | c | c | c | c |}

    \hline

    A & B & A $\land$ B & $\neg$A & $\neg$B & A $\leftrightarrow$ B & (A $\land$ B) $\lor$ ($\neg$A $\land$ $\neg$B) \\ \hline

    0 & 0 & 0 & 1 & 1 & 1 & 1 \\

    0 & 1 & 0 & 1 & 0 & 0 & 0 \\

    1 & 0 & 0 & 0 & 1 & 0 & 0 \\

    1 & 1 & 1 & 0 & 0 & 1 & 1 \\ \hline

  \end{tabular} 

\end{center}

\end{enumerate}



\section{De Morgan Equivalences}



\begin{enumerate}



  \item $\neg$(A $\land$ B) $\equiv$ $\neg$A $\lor$ $\neg$B



\begin{center}

  \begin{tabular}{| c | c | c | c |}

    \hline

    A & B & $\neg$(A $\land$ B) & $\neg$A $\lor$ $\neg$B \\ \hline

    0 & 0 & 1 & 1 \\

    0 & 1 & 1 & 1 \\ 

    1 & 0 & 1 & 1 \\

    1 & 1 & 0 & 0 \\ \hline

  \end{tabular}

\end{center}



  \item $\neg$(A $\lor$ B) $\equiv$ $\neg$A $\land$ $\neg$B



\begin{center}

  \begin{tabular}{| c | c | c | c |}

    \hline

    A & B & $\neg$(A $\lor$ B) & $\neg$A $\land$ $\neg$B \\ \hline

    0 & 0 & 1 & 1 \\

    0 & 1 & 0 & 0 \\ 

    1 & 0 & 0 & 0 \\

    1 & 1 & 0 & 0 \\ \hline

  \end{tabular}

\end{center}

\end{enumerate}



\section{Distributibivity of And and Or}

\begin{enumerate}



  \item A $\land$ (B $\lor$ C) $\equiv$ (A $\land$ B) $\lor$ (A $\land$ C)



\begin{center}

  \begin{tabular}{| c | c | c | c | c |}

    \hline

    A & B & C & A $\land$ (B $\lor$ C) & (A $\land$ B) $\lor$ (A $\land$ C) \\ \hline

    0 & 0 & 0 & 0 & 0 \\

    0 & 0 & 1 & 0 & 0 \\

    0 & 1 & 0 & 0 & 0 \\

    0 & 1 & 1 & 0 & 0 \\

    1 & 0 & 0 & 0 & 0 \\

    1 & 0 & 1 & 1 & 1 \\

    1 & 1 & 0 & 1 & 1 \\

    1 & 1 & 1 & 1 & 1 \\ \hline

  \end{tabular}

\end{center}



  \item A $\lor$ (B $\land$ C) $\equiv$ (A $\lor$ B) $\land$ (A $\lor$ C)



\begin{center}

  \begin{tabular}{| c | c | c | c | c |}

    \hline

    A & B & C & A $\lor$ (B $\land$ C) & (A $\lor$ B) $\land$ (A $\lor$ C) \\ \hline

    0 & 0 & 0 & 0 & 0 \\

    0 & 0 & 1 & 0 & 0 \\

    0 & 1 & 0 & 0 & 0 \\

    0 & 1 & 1 & 1 & 1 \\

    1 & 0 & 0 & 1 & 1 \\

    1 & 0 & 1 & 1 & 1 \\

    1 & 1 & 0 & 1 & 1 \\

    1 & 1 & 1 & 1 & 1 \\ \hline

  \end{tabular}

\end{center}

\end{enumerate}

%-------------------------------------------------------------------------------



\end{document}
