\documentclass{report}
\usepackage{graphicx}
\linespread{1.3}
\begin{document}

%-------------------------------------------------------------------------------
% TITLE PAGE
%-------------------------------------------------------------------------------

\begin{titlepage}
\newcommand{\HRule}{\rule{\linewidth}{0.5mm}}
\center
\textsc{\LARGE Imperial College London} \\[0.5cm]
\textsc{\Large Department of Computing} \\[0.5cm]
\textsc{\large Third Year Individual Project} \\[1.5cm]
\HRule \\[0.3cm]
{\huge \bfseries Equivalences Tutor} \\[0.3cm]
\HRule \\[1.5cm]

% author and supervisors
\begin{minipage}{0.4\textwidth}
\begin{flushleft} \large \emph{Author:} \\
Sahil \textsc{Jain}
\end{flushleft}
\end{minipage}~
\begin{minipage}{0.4\textwidth}
\begin{flushright} \large \emph{Supervisor:} \\
Fariba \textsc{Sadri}
\end{flushright}
\begin{flushright} \large \emph{Second marker:} \\
Ian \textsc{Hodkinson}
\end{flushright}
\end{minipage}\\[4cm]
{\large \today}\\[3cm]
\vfill
\end{titlepage}

%-------------------------------------------------------------------------------
% ABSTRACT
%-------------------------------------------------------------------------------

\begin{abstract}
% TODO
\end{abstract}


%-------------------------------------------------------------------------------
% ACKNOWLEDGEMENTS
%-------------------------------------------------------------------------------

\subsection*{\centering Acknowledgements}

%-------------------------------------------------------------------------------
% TABLE OF CONTENTS
%-------------------------------------------------------------------------------

\tableofcontents

%-------------------------------------------------------------------------------
% INTRODUCTION
%-------------------------------------------------------------------------------

\chapter{Introduction}
% what is the project about
% what did I set out to achieve, relevance, importance, difficulty
% key aspects, non-technical

\emph{Logic} refers to the study of different modes of reasoning conducted or 
assessed according to strict principles of validity. Due to Logic being one of 
the most fundamental aspects of Computer Science, it is taught to every student
pursuing a Computing degree at Imperial College London. During the first term 
in university, every Computing student is taught the Logic course, which aims to
provide the students with knowledge of the syntax and semantics of Propositional 
and Predicate logic. Students can apply this knowledge to complete equivalences
and natural deduction proofs.

A logical system is made up of three things:

\begin{enumerate}
  \item Syntax - this is the formal language specified to express different
         concepts.
  \item Semantics - this is what provides meaning to the language.
  \item Proof theory - this is a way of arguing in the language. This allows us
         to identify valid statements in the language.
\end{enumerate}

In logic, two statements are logically equivalent if they contain the same
logical content. Mendelson stated that "two statements are equivalent if they
have the same truth value in every model." This can be illustrated in the 
following example: \\ \bigskip 
Statement 1: \emph{If Sahil is a final year student, 
then he has to do an individual project} \\ \bigskip 
Statement 2: \emph{If Sahil is not 
doing an individual project, then he is not a final year student} \\ \bigskip 
As we can see, both statements have the same result in same models. When two logic 
statements are equivalent, they can be derived by each other, with the use of 
equivalences which we know to be true.


%-------------------------------------------------------------------------------
% BACKGROUND
%-------------------------------------------------------------------------------

\chapter{Background} % 10-20 pages

%-------------------------------------------------------------------------------
% MAIN BODY
%-------------------------------------------------------------------------------

\chapter{Approach and Implementation details}

\section{Modelling Logic Formulae}

The first part that had to be implemented was modelling logic formulae so that
they could be manipulated when applying equivalences. In the first year Logic course
taught by Ian Hodkinson, the students are taught the Logic formation tree.

The example given in the first year slides is:
% TODO insert the tree from first year slides

Using this formation tree, we can model logical formulae in a tree structure in
Java by creating our own data structure.

\subsection{Lexer}

The first step in modelling the logical formulae into a tree structure is lexical 
analysis. This is the process of converting a sequence of characters into a sequence
of tokens which can be parsed in the future.

As writing a lexer from scratch would have been very time consuming and tedious,
I researched about the tools which had been developed for generating lexers after
inputting the lexical grammar. The best tool which I came across was ANTLR, as it
provides a very easy to use debugger where you can load the generated code and step
through it. 

\subsubsection{Lexical Grammar}
\begin{verbatim}
lexer grammar LogicLexer;

options {
  language = Java;
}

@header {
  package eqtutor;
}

WHITESPACE:			( '\t' | ' ' | '\r' | '\n' | '\u000C' )+ { $channel = HIDDEN; };

AND	:						'&';
OR	:						'|';
IFTHEN	:				'->';
IFF 	:					'<>';
NOT 	:					'!';

LPAREN  :				'(';
RPAREN  :				')';

ID	:						('A'..'Z'|'a'..'z') ('A'..'Z'|'a'..'z'|'_')*;
\end{verbatim}

\subsection{Parser}

Once the formulae has been tokenised by the lexer, it then has to be parsed. Parsing
is the process where the tokens are used to build a data structure, which is usually
a hierarchical structure. This data structure gives a structural representation
of the input.

ANTLR provides both lexer and parser generators. Due to having the ability of doing both
of these steps using one tool which I had become familiar with, I decided to use the
ANTLR parser generator.


\subsubsection{Parser Grammar}

\begin{verbatim}
parser grammar LogicParser;

options {
  tokenVocab = LogicLexer;
}

@header {
  package eqtutor;

  import java.util.LinkedList;
  import java.util.List;
  import AST.*;
}

@members {
  private boolean hasFoundError = false;

  public void displayRecognitionError(String[] tokenNames, RecognitionException e) {
    hasFoundError = true;
  }

  public boolean hasFoundError() {
    return this.hasFoundError;
  }
}

program returns [AST tree]
  : e = iffexpr {$tree = new AST(new ASTProgramNode($e.node));} EOF
  ;

iffexpr returns [ASTPropositionalNode node]
  : ifthen = ifexpr {$node = $ifthen.node;} 
	  (IFF iff = iffexpr {$node = new ASTIffNode($ifthen.node, $iff.node);})*
  ;
  	
ifexpr returns [ASTPropositionalNode node]
  : or = orexpr {$node = $or.node;} 
		(IFTHEN ifthen = ifexpr {$node = new ASTIfThenNode($or.node, $ifthen.node);})*
  ;

orexpr returns [ASTPropositionalNode node]
  : and = andexpr {$node = $and.node;} 
		(OR or = orexpr {$node = new ASTOrNode($and.node, $or.node);})*
  ;

andexpr returns [ASTPropositionalNode node]
  : not = notexpr {$node = $not.node;} 
		(AND and = andexpr {$node = new ASTAndNode($not.node, $and.node);})*
  ;

notexpr returns [ASTPropositionalNode node]
  : NOT not = notexpr {$node = new ASTNotNode($not.node);}
  | id = identifier {$node = $id.node;}
  ;

identifier returns [ASTPropositionalNode node]
  : ID {$node = new ASTIdentifierNode($ID.text);}
  | LPAREN iffexpr RPAREN {$node = $iffexpr.node;}
  ;
\end{verbatim}

\section{Abstract Syntax Tree}

The Abstract Syntax Tree (AST) is the

%-------------------------------------------------------------------------------
% EVALUATION
%-------------------------------------------------------------------------------

\chapter{Evaluation}

%-------------------------------------------------------------------------------
% CONCLUSIONS AND FUTURE WORK
%-------------------------------------------------------------------------------

\chapter{Conclusions and Future Work}
\section{Learning outcomes}
\section{Potential improvements}
\section{Potential extensions}

%-------------------------------------------------------------------------------
% BIBLIOGRAPHY
%-------------------------------------------------------------------------------

\begin{thebibliography}{9}
%\bibitem{lamport94}
% Leslie Lamport,
% \emph{\LaTeX: A Document Preparation System}.
% Addison Wesley, Massachusetts,
% 2nd Edition,
% 1994.
\end{thebibliography}

%-------------------------------------------------------------------------------
% APPENDIX
%-------------------------------------------------------------------------------

\appendix
% \chapter{Short demo}
% \chapter{Program Listings}

%-------------------------------------------------------------------------------

\end{document}
